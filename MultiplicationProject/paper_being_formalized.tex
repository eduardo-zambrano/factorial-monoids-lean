\documentclass[11pt]{amsart}

\usepackage[a4paper,margin=1in]{geometry}
\usepackage{amsmath,amssymb,amsthm,mathtools}
\usepackage{enumitem}
\usepackage{hyperref}

\newtheorem{theorem}{Theorem}[section]
\newtheorem{lemma}[theorem]{Lemma}
\newtheorem{proposition}[theorem]{Proposition}
\newtheorem{corollary}[theorem]{Corollary}

\theoremstyle{definition}
\newtheorem{definition}{Definition}
\newtheorem*{axiom}{Axiom}
\newtheorem*{property}{Property}
\newtheorem{example}{Example}
\newtheorem{remark}{Remark}

\newcommand{\N}{\mathbb{N}}
\newcommand{\Nzero}{\mathbb{N}_0}
\newcommand{\Z}{\mathbb{Z}}
\newcommand{\PP}{\mathbb{P}}

\title{Characterizing Factorial Monoids via Labeled Factorization Counts}

\author{Eduardo Zambrano}
\address{California Polytechnic State University, San Luis Obispo, CA 93407, USA}
\email{ezambran@calpoly.edu}
\date{\today}

\subjclass[2020]{Primary 20M14; Secondary 11A51, 11A25, 05A15}
\keywords{factorial monoid; free commutative monoid; labeled factorization counts; unique factorization; reduced atomic monoid; stars-and-bars identity; coprime independence}

\begin{document}

\maketitle

\begin{abstract}
Let $M$ be a reduced atomic commutative monoid. We introduce the \emph{labeled $k$-factorization counts} $F_k(m) = \#\{(m_1,\ldots,m_k) \in M^k : m_1 \cdots m_k = m\}$ and give a purely combinatorial characterization of when $M$ is factorial (i.e., a free commutative monoid). Specifically, we impose three axioms on the factorization structure: \textup{(PP-D)} powers of each atom are distinct, \textup{(CFI)} coprime parts factor independently (a bijection condition on factorization sets), and \textup{(CPL)} pairwise coprime tuples exist in every length. We prove that \textup{(PP-D)} together with \textup{(CFI)} forces the local and global stars-and-bars identities and establishes that $M$ is factorial; adding \textup{(CPL)} forces the atom set to be countably infinite, yielding $M \cong (\N, \times)$. We demonstrate sharpness by constructing explicit reduced atomic monoids in which any two of the three axioms hold but the third fails.
\end{abstract}

%=============================================================================
\section{Introduction}\label{sec:intro}
%=============================================================================

The theory of factorization in commutative monoids and integral domains has developed into a rich area of algebra, with central objects of study including sets of lengths, elasticity, catenary degree, and the structure of Krull monoids; see the monograph of Geroldinger and Halter-Koch~\cite{GeroldingerHalterKoch2006} and the survey~\cite{GeroldingerZhong2020}. Much of this theory is devoted to understanding and classifying the \emph{failure} of unique factorization---measuring how far a given monoid departs from being factorial.

In this paper we take a complementary perspective: rather than studying non-unique factorization, we ask what \emph{combinatorial conditions on factorization counts} force a reduced atomic commutative monoid to be factorial. Our main theorem shows that three natural counting axioms on labeled factorizations 
characterize factorial monoids among reduced atomic commutative monoids. Our main objects are the \emph{labeled $k$-factorization counts}
\[
F_k(m) := \#\bigl\{(m_1,\ldots,m_k) \in M^k : m_1 \cdots m_k = m\bigr\},
\]
which count ordered $k$-tuples (with the identity element allowed in slots) whose product is $m$. For the multiplicative monoid $(\N, \times)$, these counts satisfy the classical stars-and-bars formula
\[
F_k(n) = \prod_{p} \binom{e_p(n) + k - 1}{k - 1},
\]
where $e_p(n)$ is the exponent of the prime $p$ in the factorization of $n$.

We reverse this classical computation: starting from an abstract reduced atomic commutative monoid $M$ with atom set $P$, we impose three transparent axioms---\textup{(PP-D)}, \textup{(CFI)}, and \textup{(CPL)}---and prove that these force the stars-and-bars formula to hold, first locally (for powers of each atom) and then globally (across coprime parts). From the master counting formula we derive that the $p$-adic valuations are additive homomorphisms, establishing that $M$ is the free commutative monoid on $P$. The additional axiom \textup{(CPL)} then forces $|P| = \aleph_0$, yielding $M \cong (\N, \times)$.

\subsection*{Related work}

On the combinatorial side, ordered factorizations of integers have been studied extensively, typically counting factorizations with all factors $\ge 2$ and often unordered; see, e.g., \cite{OrderedFact2016}. Our $F_k$ counts differ in that they are labeled (ordered tuples with the identity allowed); for factorial monoids such as $(\mathbb{N}, \times)$, they reduce locally to the classical stars-and-bars identity $F_k(p^e) = \binom{e+k-1}{k-1}$; see~\cite{GoemansGenFunc}.



Our axiom \textup{(CFI)} is the purely combinatorial counterpart of Euler-product factorization for Dirichlet series with multiplicative coefficients: multiplicativity on coprime inputs is equivalent to factorization into local prime-power series; see \cite[Chs.~2--3]{ApostolANT} and \cite[\S I.2--I.4]{Tenenbaum}. We use this only at the level of counts; no analytic hypotheses are required.

\subsection*{Outline}

Section~\ref{sec:prelim} establishes notation and recalls standard definitions from factorization theory. Section~\ref{sec:axioms} introduces our three axioms as factorization properties. Section~\ref{sec:strategy} gives a proof outline. Sections~\ref{sec:local-purity}--\ref{sec:master} contain the main arguments: deriving local purity, the local stars-and-bars formula, global multiplicativity, and the master counting formula. Section~\ref{sec:main} states and proves the main theorem. Section~\ref{sec:sharpness} demonstrates sharpness via counterexamples. Section~\ref{sec:conclusion} discusses extensions, and the Appendix records a technical reduction lemma.

%=============================================================================
\section{Preliminaries}\label{sec:prelim}
%=============================================================================

We recall standard terminology from factorization theory; see \cite{GeroldingerHalterKoch2006, GeroldingerZhong2020} for comprehensive treatments.

\begin{definition}[Monoid terminology]\label{def:monoid-terms}
Let $(M, \cdot, 1)$ be a commutative monoid with identity $1$.
\begin{enumerate}[label=(\roman*), itemsep=2pt]
\item An element $u \in M$ is a \textbf{unit} if there exists $v \in M$ with $uv = 1$.
\item $M$ is \textbf{reduced} if $1$ is the only unit.
\item A non-unit $a \in M$ is an \textbf{atom} (or \textbf{irreducible}) if $a = bc$ implies $\{b,c\} = \{1,a\}$.
\item $M$ is \textbf{atomic} if every non-unit can be written as a finite product of atoms.
\item $M$ is \textbf{factorial} (or a \textbf{unique factorization monoid}) if every non-unit has a unique factorization into atoms, up to order.
\end{enumerate}
\end{definition}

\begin{remark}\label{rem:free-comm}
A commutative monoid is factorial if and only if it is isomorphic to a free commutative monoid $\bigoplus_{p \in P} \Nzero$ for some set $P$ (the atoms). In this case, $P$ is uniquely determined up to bijection; see \cite[Ch.~1]{GeroldingerHalterKoch2006}.
\end{remark}

Throughout this paper, $(M, \cdot, 1)$ denotes a reduced atomic commutative monoid with atom set $P$. We write $M^* = M \setminus \{1\}$ for the set of non-units. We do not assume cancellativity; it will follow from our axioms (see Remark~\ref{rem:cancellativity}). 

\begin{definition}[Exponentiation]\label{def:exp}
For $p \in P$ and $e \in \Nzero$, define $p^0 := 1$ and $p^{e+1} := p^e \cdot p$. Associativity gives
\begin{equation}\label{eq:exp-add}
p^a \cdot p^b = p^{a+b} \qquad (a, b \in \Nzero).
\end{equation}
We write $\langle p \rangle = \{p^e : e \ge 0\}$ for the submonoid generated by $p$.
\end{definition}

For $m \in M$, let $\omega(m)$ denote the number of distinct atoms dividing $m$.

\begin{definition}[Divisibility and coprimality]\label{def:div-coprime}
For $p \in P$ and $m \in M$, write $p \mid m$ if $m = p \cdot m'$ for some $m' \in M$. Two elements $x, y \in M$ are \textbf{coprime} if no atom divides both. The \textbf{support} of $m$, denoted $\mathrm{supp}(m)$, is the set of atoms dividing $m$.
\end{definition}

\begin{definition}[Labeled factorization counts]\label{def:counts}
For $k \in \N$ and $m \in M$, define the \textbf{set of labeled $k$-factorizations} as
\[
\mathcal{F}_k(m) := \bigl\{(m_1, \ldots, m_k) \in M^k : m_1 \cdots m_k = m\bigr\},
\]
and the \textbf{labeled $k$-factorization count} as $F_k(m) := |\mathcal{F}_k(m)|$. We allow the identity $1$ in slots and count \emph{ordered} $k$-tuples.
\end{definition}

\begin{remark}[Comparison with lengths sets]\label{rem:lengths}
The labeled counts $F_k$ differ from the ``lengths sets'' studied in classical factorization theory \cite{GeroldingerHalterKoch2006}, which count \emph{unordered} factorizations of a given element into non-units. In the usual language of factorization theory, one works with BF-monoids and their associated system of sets of lengths $\{\mathsf L(m) : m \in M\}$, where $\mathsf L(m)$ collects all possible factorization lengths of $m$. Our counts are finer: they are ordered, labeled $k$-factorization counts which remember not only the possible lengths but also how many factorizations of each length exist, reducing locally to the combinatorial stars-and-bars formula in the factorial case.
\end{remark}

\begin{example}[The multiplicative monoid of positive integers]\label{ex:ordinary}
Let $M = (\N, \times, 1)$ with $P = \PP$ (the set of primes). This is the prototypical factorial monoid. For $n = \prod_p p^{e_p(n)}$, the stars-and-bars formula gives
\[
F_k(n) = \prod_{p} \binom{e_p(n) + k - 1}{k - 1}.
\]
\end{example}

\begin{example}[Relabeling of primes]\label{ex:twisted}
Fix a bijection $\sigma : \PP \to \PP$ and extend it multiplicatively: for $n = \prod_p p^{v_p(n)}$, set $\sigma(n) := \prod_p \sigma(p)^{v_p(n)}$. Define a new operation on $\N$ by
\[
n \star m := \sigma^{-1}\bigl(\sigma(n) \cdot \sigma(m)\bigr).
\]
Then $(\N, \star, 1)$ is a reduced atomic commutative monoid with atoms $P = \sigma^{-1}(\PP)$. The map $\sigma : (\N, \star) \to (\N, \times)$ is an isomorphism, so $(\N, \star)$ is factorial. This example shows that factorial monoids isomorphic to $(\N, \times)$ can ``look different'' while having identical factorization-theoretic properties.
\end{example}

\begin{example}[The Peano monoid]\label{ex:peano}
Define an operation on $\N$ by $x \star y := x + y - 1$. Then $(\N, \star, 1)$ is a reduced atomic commutative monoid with a single atom $P = \{2\}$, and $2^{\star e} = e + 1$. This monoid is factorial (isomorphic to $(\Nzero, +)$) but has only one atom, so it cannot be isomorphic to $(\N, \times)$.
\end{example}

%=============================================================================
\section{The Axioms}\label{sec:axioms}
%=============================================================================

We now introduce three factorization properties that a reduced atomic commutative monoid may or may not satisfy.

\begin{axiom}[\textbf{PP-D}: Powers of atoms are distinct]
For every $p \in P$, the map $\Nzero \to M$, $e \mapsto p^e$, is injective. Equivalently, $p^a = p^b$ implies $a = b$.
\end{axiom}

This axiom ensures that powers of each atom form an infinite sequence $1, p, p^2, p^3, \ldots$ with no collapses. Without it, a relation like $p^2 = p$ would create ``extra'' factorizations, inflating the counts beyond the stars-and-bars prediction. More generally, when all powers of $p$ are distinct, every two-factorization $p^e = m_1 \cdot m_2$ forces unique exponents $f_1, f_2$ with $f_1 + f_2 = e$, giving exactly $e + 1$ possibilities.

\begin{axiom}[\textbf{CFI}: Coprime parts factor independently]
If $x, y \in M$ are coprime, then the coordinatewise multiplication map
\[
\mu_2 : \mathcal{F}_2(x) \times \mathcal{F}_2(y) \longrightarrow \mathcal{F}_2(x \cdot y), \qquad \bigl((x_1, x_2), (y_1, y_2)\bigr) \mapsto (x_1 y_1, x_2 y_2)
\]
is a bijection.
\end{axiom}

This axiom asserts that when $x$ and $y$ have disjoint atom supports, their factorizations combine independently via coordinatewise multiplication. Without it, spurious factorizations can arise---for instance, if $u \cdot v = p \cdot q$ for distinct atoms $u, v, p, q$, then the product $p \cdot q$ would have a two-factorization $(u, v)$ not arising from coordinatewise assembly (surjectivity fails). Conversely, if $a \cdot c = b \cdot d$ for distinct atoms $a, b, c, d$, the distinct pairs $((a, b), (c, d))$ and $((b, a), (d, c))$ would both map to the same factorization of $(a \cdot b) \cdot (c \cdot d)$ (injectivity fails). \textup{(CFI)} rules out both pathologies.

\begin{remark}\label{rem:CFI-multiplicativity}
Taking cardinalities, \textup{(CFI)} implies $F_2(x \cdot y) = F_2(x) \cdot F_2(y)$ for coprime $x, y$---the counts are multiplicative on coprime inputs. But \textup{(CFI)} is stronger: it asserts a bijection at the level of factorization \emph{sets}, not just equality of cardinalities.
\end{remark}

\begin{axiom}[\textbf{CPL}: Coprime tuples come in every length]
For every $r \in \N$, there exist non-units $m_1, \ldots, m_r \in M^*$ that are pairwise coprime.
\end{axiom}

This axiom ensures an unbounded supply of ``independent building blocks.'' In the spirit of Mercer's deficit-of-factorizations argument \cite{Mercer2017}, it precludes the combinatorial bottleneck that would occur with only finitely many atoms: with $|P| < \infty$, squarefree products are bounded by $2^{|P|} - 1$ possibilities, preventing arbitrarily long coprime tuples.

We also define a property that will be \emph{derived} from the axioms:

\begin{property}[\textbf{PP-P}: Prime powers are factorially closed]
For every $p \in P$, if $x \cdot y \in \langle p \rangle$, then $x, y \in \langle p \rangle$.
\end{property}

This property says each atom $p$ defines an ``isolated tower'': if a product lands in $\langle p \rangle$, both factors must already be powers of $p$. It prevents cross-prime contamination such as $p^2 = q \cdot r$ for distinct atoms.

%=============================================================================
\section{Proof Strategy}\label{sec:strategy}
%=============================================================================

We outline how the axioms combine to yield the main theorem.

\begin{enumerate}[label=\textbf{(S\arabic*)}, leftmargin=3em]
\item \textbf{CFI implies PP-P} (Section~\ref{sec:local-purity}): Using blockwise independence from \textup{(CFI)}, we show that each $\langle p \rangle$ is factorially closed.

\item \textbf{Local stars-and-bars} (Section~\ref{sec:local}): From \textup{(PP-D)} and \textup{(PP-P)}, we prove $F_k(p^e) = \binom{e+k-1}{k-1}$ for all atoms $p$ and exponents $e$.

\item \textbf{Global multiplicativity} (Section~\ref{sec:global}): From \textup{(CFI)}, we show the bijection $\mu_2$ extends to $\mu_k$ for all $k$, giving $F_k(x \cdot y) = F_k(x) \cdot F_k(y)$ for coprime $x, y$.

\item \textbf{Master formula} (Section~\ref{sec:master}): Combining the local and global steps yields
\[
F_k(m) = \prod_{p \in P} \binom{v_p(m) + k - 1}{k - 1},
\]
where $v_p(m) := \max\{e \ge 0 : p^e \mid m\}$.

\item \textbf{Additivity of valuations}: The master formula forces $v_p(x \cdot y) = v_p(x) + v_p(y)$, making each $v_p$ a monoid homomorphism.

\item \textbf{Factorial structure}: The map $\Phi : \bigoplus_{p \in P} \Nzero \to M$, $(e_p)_p \mapsto \prod_p p^{e_p}$, is a monoid isomorphism. Thus $M$ is factorial.

\item \textbf{Cardinality of $P$}: Axiom \textup{(CPL)} forces $|P| = \aleph_0$. Choosing a bijection $\iota : P \to \PP$ yields $M \cong (\N, \times)$.
\end{enumerate}

\begin{remark}
It suffices to verify \textup{(CFI)} only for pairs of prime powers; we build up first to one prime power against a squarefree product of prime powers, then to arbitrary coprime inputs. We record this reduction in the Appendix.
\end{remark}

%=============================================================================
\section{Local Purity from Coprime Independence}\label{sec:local-purity}
%=============================================================================

We show that \textup{(CFI)} implies the property \textup{(PP-P)}.

\begin{definition}[Blockwise disjointness]\label{def:blockwise}
A family $\{(x^{(j)}, y^{(j)})\}_{j=1}^m$ is \textbf{blockwise disjoint} if
\[
\bigl(\mathrm{supp}(x^{(i)}) \cup \mathrm{supp}(y^{(i)})\bigr) \cap \bigl(\mathrm{supp}(x^{(j)}) \cup \mathrm{supp}(y^{(j)})\bigr) = \varnothing \quad \text{for all } i \neq j.
\]
\end{definition}

\begin{lemma}[Blockwise CFI for two factors]\label{lem:blockwise-2}
Assume \textup{(CFI)} holds (bijectivity for $k = 2$ across coprime inputs). Let $\{(x^{(j)}, y^{(j)})\}_{j=1}^m$ be a blockwise disjoint family with each $x^{(j)}$ and $y^{(j)}$ coprime. Set $X := \prod_{j=1}^m x^{(j)}$ and $Y := \prod_{j=1}^m y^{(j)}$. Define
\[
\Theta : \prod_{j=1}^m \bigl(\mathcal{F}_2(x^{(j)}) \times \mathcal{F}_2(y^{(j)})\bigr) \longrightarrow \mathcal{F}_2(X \cdot Y)
\]
by
\[
\Theta\Bigl(\bigl((a^{(j)}, b^{(j)}), (c^{(j)}, d^{(j)})\bigr)_{j=1}^m\Bigr) := \Bigl(\prod_{j=1}^m a^{(j)} c^{(j)},\, \prod_{j=1}^m b^{(j)} d^{(j)}\Bigr).
\]
Then $\Theta$ is a bijection.
\end{lemma}

\begin{proof}
Induction on $m$. For $m = 1$, this is \textup{(CFI)}. For the step $m \to m+1$, write $X = X' \cdot x^{(m+1)}$ and $Y = Y' \cdot y^{(m+1)}$ with $\mathrm{supp}(X' \cdot Y')$ disjoint from $\mathrm{supp}(x^{(m+1)} \cdot y^{(m+1)})$. Apply \textup{(CFI)} to the coprime pair $(X', Y')$ and $(x^{(m+1)}, y^{(m+1)})$ to combine the two-factor bijection given by the induction hypothesis (for $(X', Y')$) with the two-factor bijection for $(x^{(m+1)}, y^{(m+1)})$. Associativity and commutativity of the monoid operation and blockwise disjointness ensure coherence and invertibility.
\end{proof}

\begin{lemma}[Blockwise CFI for $k$ factors]\label{lem:blockwise-k}
Assume \textup{(CFI)} holds and let $\{(x^{(j)}, y^{(j)})\}_{j=1}^m$ be blockwise disjoint with each pair coprime. Fix $\ell \in \{1, \ldots, m\}$ and set $X := \prod_{j=1}^m x^{(j)}$, $Y := \prod_{j=1}^m y^{(j)}$. There is a canonical bijection
\[
\prod_{j \neq \ell} \bigl(\mathcal{F}_2(x^{(j)}) \times \mathcal{F}_2(y^{(j)})\bigr) \times \mathcal{F}_k(x^{(\ell)}) \times \mathcal{F}_k(y^{(\ell)}) \;\xrightarrow{\;\cong\;}\; \mathcal{F}_k(X \cdot Y)
\]
obtained by applying Lemma~\ref{lem:blockwise-2} to all blocks $j \neq \ell$ (to produce a two-factorization of the product over $j \neq \ell$), and then using the $k$-factor bijection on the single coprime pair $(x^{(\ell)}, y^{(\ell)})$ to distribute the $k$ coordinates.
\end{lemma}

\begin{proof}
Combine Lemma~\ref{lem:blockwise-2} (two-factor assembly across the blocks $j \neq \ell$) with the bijection on the (coprime) pair $(x^{(\ell)}, y^{(\ell)})$. Blockwise disjointness again gives coherence and invertibility.
\end{proof}

\begin{proposition}[CFI implies PP-P]\label{prop:CFI-implies-PPP}
Assume \textup{(CFI)}. Then for each $p \in P$, the submonoid $\langle p \rangle$ is factorially closed: if $x \cdot y \in \langle p \rangle$, then $x, y \in \langle p \rangle$.
\end{proposition}

\begin{proof}
Fix $p \in P$ and suppose $x \cdot y \in \langle p \rangle$. Choose factorizations of $x$ and $y$ into atoms and write
\[
x = p^a \cdot x_{\mathrm{pf}}, \qquad y = p^b \cdot y_{\mathrm{pf}},
\]
where $\mathrm{supp}(x_{\mathrm{pf}}) \cap \{p\} = \mathrm{supp}(y_{\mathrm{pf}}) \cap \{p\} = \varnothing$ (here ``pf'' stands for ``$p$-free''). Partition $x_{\mathrm{pf}}$ and $y_{\mathrm{pf}}$ as products of blocks
\[
x_{\mathrm{pf}} = \prod_{j=1}^r u^{(j)}, \qquad y_{\mathrm{pf}} = \prod_{i=1}^s v^{(i)},
\]
such that the supports of the blocks are pairwise disjoint and also disjoint from $\{p\}$ (group atoms by distinct primes). Then the family
\[
\bigl(p^a, p^b\bigr), \quad \bigl(u^{(1)}, 1\bigr), \ldots, \bigl(u^{(r)}, 1\bigr), \quad \bigl(1, v^{(1)}\bigr), \ldots, \bigl(1, v^{(s)}\bigr)
\]
is blockwise disjoint, and each pair in it is coprime.

Apply Lemma~\ref{lem:blockwise-k} with $\ell$ the $p$-block and $k := r + s + 1$. Choose on the $p$-block the $k$-factorization that puts the entire $p$-part in the first coordinate:
\[
\bigl(p^{a+b}, 1, \ldots, 1\bigr) \in \mathcal{F}_k(p^a) \times \mathcal{F}_k(p^b).
\]
By bijectivity, this has a unique preimage under the assembled blockwise map. Inspecting the image shows that every coordinate equals a $p$-power (since $x \cdot y \in \langle p \rangle$). In particular, for each block $(u^{(j)}, 1)$ (resp.\ $(1, v^{(i)})$), the only way its contribution can leave all coordinates inside $\langle p \rangle$ is if $u^{(j)} = 1$ (resp.\ $v^{(i)} = 1$); otherwise some coordinate would carry a non-$p$ atom. Thus $x_{\mathrm{pf}} = y_{\mathrm{pf}} = 1$, hence $x = p^a$ and $y = p^b$, proving \textup{(PP-P)}.
\end{proof}

%=============================================================================
\section{Local Characterization on Prime Powers}\label{sec:local}
%=============================================================================

\begin{lemma}[Unique factorization within prime powers]\label{lem:pp-unique}
Assume \textup{(PP-D)} and \textup{(PP-P)}. If $p^e = x \cdot y$, then $x = p^a$ and $y = p^{e-a}$ for a unique $a \in \{0, \ldots, e\}$.
\end{lemma}

\begin{proof}
By \textup{(PP-P)}, $x, y \in \langle p \rangle$, say $x = p^a$, $y = p^b$. Then $p^e = p^{a+b}$, and \textup{(PP-D)} forces $e = a + b$. Uniqueness is immediate.
\end{proof}

\begin{definition}[Valuation]\label{def:valuation}
Under \textup{(PP-D)} and \textup{(PP-P)}, for $p \in P$ and $m \in M$, define $v_p(m) := \max\{e \ge 0 : p^e \mid m\}$, where $p^e \mid m$ means $m = p^e \cdot m'$ for some $m'$. By Lemma~\ref{lem:pp-unique}, this maximum is well-defined.
\end{definition}

\begin{theorem}[Local stars-and-bars]\label{thm:local-sb}
Under \textup{(PP-D)} and \textup{(PP-P)}, for all $p \in P$, $e \ge 0$, and $k \ge 1$,
\[
F_k(p^e) = \binom{e + k - 1}{k - 1}.
\]
\end{theorem}

\begin{proof}
By associativity \eqref{eq:exp-add}, for each weak composition $e = e_1 + \cdots + e_k$ we have $p^{e_1} \cdots p^{e_k} = p^e$, so $F_k \ge \binom{e+k-1}{k-1}$. Conversely, any $(m_1, \ldots, m_k)$ with product $p^e$ lies in $\langle p \rangle^k$ by \textup{(PP-P)}, hence $m_i = p^{e_i}$ and $e_1 + \cdots + e_k = e$ by \textup{(PP-D)}. Counting weak compositions gives the upper bound.
\end{proof}

%=============================================================================
\section{Global Multiplicativity from CFI}\label{sec:global}
%=============================================================================

We first propagate the bijection to all $k$.

\begin{lemma}[CFI extends to all $k$]\label{lem:CFI-k}
Assume \textup{(CFI)} for two factors. If $x, y$ are coprime, then for every $k \ge 2$ the coordinatewise map
\[
\mu_k : \mathcal{F}_k(x) \times \mathcal{F}_k(y) \longrightarrow \mathcal{F}_k(x \cdot y), \qquad \bigl((x_i)_{i=1}^k, (y_i)_{i=1}^k\bigr) \mapsto (x_i y_i)_{i=1}^k
\]
is a bijection.
\end{lemma}

\begin{proof}
Let $x, y$ be coprime. We argue by induction on $k \ge 2$.

\smallskip
\emph{Base $k = 2$.} This is exactly \textup{(CFI)}.

\smallskip
\emph{Induction step $k \to k+1$: surjectivity.}
Let $(w_1, \ldots, w_{k+1}) \in \mathcal{F}_{k+1}(x \cdot y)$. Set $u := w_2 \cdots w_{k+1}$, so $(w_1, u) \in \mathcal{F}_2(x \cdot y)$. By \textup{(CFI)}, there are unique two-factorizations $x = x_1 \cdot x'$, $y = y_1 \cdot y'$ with
\[
w_1 = x_1 \cdot y_1, \qquad u = x' \cdot y'.
\]
Since $x$ and $y$ are coprime, the induction hypothesis applied to $(x', y')$ and the $k$-tuple $(w_2, \ldots, w_{k+1}) \in \mathcal{F}_k(x' \cdot y')$ yields $(x_2, \ldots, x_{k+1}) \in \mathcal{F}_k(x')$ and $(y_2, \ldots, y_{k+1}) \in \mathcal{F}_k(y')$ such that $w_i = x_i \cdot y_i$ for $i = 2, \ldots, k+1$. Together with $x_1, y_1$ this gives a preimage under $\mu_{k+1}$.

\smallskip
\emph{Induction step $k \to k+1$: injectivity.}
Suppose $\mu_{k+1}\bigl((x_i), (y_i)\bigr) = \mu_{k+1}\bigl((x'_i), (y'_i)\bigr)$, i.e.,
\[
x_i \cdot y_i = x'_i \cdot y'_i \qquad (i = 1, \ldots, k+1).
\]
Set $X' := x_2 \cdots x_{k+1}$, $Y' := y_2 \cdots y_{k+1}$ and similarly $X'^* := x'_2 \cdots x'_{k+1}$, $Y'^* := y'_2 \cdots y'_{k+1}$. Then
\[
(x_1, X') \in \mathcal{F}_2(x), \quad (y_1, Y') \in \mathcal{F}_2(y), \quad (x'_1, X'^*) \in \mathcal{F}_2(x), \quad (y'_1, Y'^*) \in \mathcal{F}_2(y),
\]
and
\[
x_1 \cdot y_1 = x'_1 \cdot y'_1, \qquad X' \cdot Y' = X'^* \cdot Y'^*.
\]
Apply \textup{(CFI)}-injectivity to the equal two-factorizations $(x_1, X') \cdot (y_1, Y') = (x'_1, X'^*) \cdot (y'_1, Y'^*)$ of $x \cdot y$ to obtain
\[
x_1 = x'_1, \quad y_1 = y'_1, \quad X' = X'^*, \quad Y' = Y'^*.
\]
Now $X' = X'^*$ and $Y' = Y'^*$ mean $(x_2, \ldots, x_{k+1}) \in \mathcal{F}_k(x')$ and $(y_2, \ldots, y_{k+1}) \in \mathcal{F}_k(y')$ map to the same $k$-tuple as $(x'_2, \ldots, x'_{k+1}) \in \mathcal{F}_k(x')$ and $(y'_2, \ldots, y'_{k+1}) \in \mathcal{F}_k(y')$. By the induction hypothesis (injectivity for $k$), we conclude $x_i = x'_i$ and $y_i = y'_i$ for $i = 2, \ldots, k+1$. Together with $x_1 = x'_1$, $y_1 = y'_1$, this proves injectivity.

\smallskip
Thus $\mu_{k+1}$ is bijective, completing the induction.
\end{proof}

\begin{proposition}[Coprime multiplicativity of counts]\label{prop:coprime-mult}
Under \textup{(CFI)}, if $x$ and $y$ are coprime, then $F_k(x \cdot y) = F_k(x) \cdot F_k(y)$ for all $k \ge 1$.
\end{proposition}

\begin{proof}
Take cardinalities in Lemma~\ref{lem:CFI-k}. The case $k = 1$ is trivial.
\end{proof}

\begin{corollary}[Squarefree diagnostic]\label{cor:squarefree}
Assume \textup{(PP-D)} and \textup{(CFI)}. If $m$ is squarefree (i.e., $v_p(m) \le 1$ for all $p$) with $\omega(m)$ distinct atoms, then $F_k(m) = k^{\omega(m)}$.
\end{corollary}

\begin{proof}
By Theorem~\ref{thm:local-sb}, $F_k(p) = k$ for each atom $p$. Since the atom factors of a squarefree element are pairwise coprime, Proposition~\ref{prop:coprime-mult} gives $F_k(m) = \prod_{p \mid m} F_k(p) = k^{\omega(m)}$.
\end{proof}

%=============================================================================
\section{Master Formula and Structural Consequences}\label{sec:master}
%=============================================================================

We now combine the local and global steps.

\begin{lemma}[Primewise decomposition]\label{lem:primewise}
Assume \textup{(PP-D)}, \textup{(PP-P)}, and atomicity. For each $m \in M$, there is a finite set $S \subset P$ and exponents $e_p = v_p(m) \in \Nzero$ such that
\[
m = \Bigl(\prod_{p \in S} p^{e_p}\Bigr) \cdot r, \qquad \text{with } p \nmid r \text{ for all } p \in S.
\]
Moreover, taking $S$ to be the set of atoms that appear in \emph{some} factorization of $m$, the leftover satisfies $r = 1$, hence
\[
m = \prod_{p \in S} p^{v_p(m)}.
\]
\end{lemma}

\begin{proof}
By atomicity, $m$ admits a factorization into atoms; let $S$ be the finite set of distinct atoms occurring in one such factorization. For each $p \in S$, define $e_p := v_p(m)$ and write $m = p^{e_p} \cdot r_p$ with $p \nmid r_p$ by the definition of $v_p$ and \textup{(PP-P)}. Extract these $p$-power factors for all $p \in S$ in any order (using associativity and commutativity), obtaining
\[
m = \Bigl(\prod_{p \in S} p^{e_p}\Bigr) \cdot r, \qquad \text{with } p \nmid r \text{ for all } p \in S,
\]
because \textup{(PP-P)} preserves $p$-freeness when multiplying by $q$-powers with $q \neq p$.

If $r \neq 1$, atomicity applied to $r$ yields an atom $q \mid r$. Then $q \mid m$, so $q \in S$, contradicting $q \nmid r$. Hence $r = 1$, as claimed.
\end{proof}

\begin{theorem}[Master counting formula]\label{thm:master}
Assume \textup{(PP-D)} and \textup{(CFI)}. For any $m \in M$ and $k \ge 1$,
\[
F_k(m) = \prod_{p \in P} \binom{v_p(m) + k - 1}{k - 1}.
\]
\end{theorem}

\begin{proof}
By Proposition~\ref{prop:CFI-implies-PPP}, \textup{(CFI)} implies \textup{(PP-P)}, so Lemma~\ref{lem:primewise} and Theorem~\ref{thm:local-sb} apply. By Lemma~\ref{lem:primewise}, $m = \prod_{p : v_p(m) > 0} p^{v_p(m)}$. By Theorem~\ref{thm:local-sb} and Proposition~\ref{prop:coprime-mult},
\[
F_k(m) = \prod_{p : v_p(m) > 0} F_k(p^{v_p(m)}) = \prod_{p : v_p(m) > 0} \binom{v_p(m) + k - 1}{k - 1}.
\]
Extending the product over all $p \in P$ (atoms with $v_p(m) = 0$ contribute $\binom{k-1}{k-1} = 1$) yields the formula.
\end{proof}

\begin{remark}
By Lemma~\ref{lem:app-pp-testing} in the Appendix, the prime-power testing hypothesis upgrades to full \textup{(CFI)}; hence Theorem~\ref{thm:master} remains valid verbatim in that weaker regime.
\end{remark}

With the master formula in place, we now pass from counting to structure by extracting additivity of the local exponents via the valuation.

\begin{proposition}[Additivity of valuations]\label{prop:val-additive}
Assume \textup{(PP-D)} and \textup{(CFI)}. For every $p \in P$ and all $x, y \in M$,
\[
v_p(x \cdot y) = v_p(x) + v_p(y).
\]
\end{proposition}

\begin{proof}
Set $a := v_p(x)$ and $b := v_p(y)$. By definition of valuation, we can write
\[
x = p^a \cdot x' \quad \text{and} \quad y = p^b \cdot y'
\]
where $v_p(x') = v_p(y') = 0$ (meaning neither $x'$ nor $y'$ is divisible by $p$).

Define $s := p^{a+b}$ and $t := x' \cdot y'$. Then $x \cdot y = s \cdot t$, and since $v_p(t) = 0$ (by \textup{(PP-P)}, if $p$ divided $t = x' \cdot y'$, then $p$ would divide $x'$ or $y'$, contradicting $v_p(x') = v_p(y') = 0$), the elements $s$ and $t$ are coprime.

\smallskip
\noindent\textbf{Lower bound: $v_p(x \cdot y) \ge a + b$.}

Since $x \cdot y = s \cdot t$ where $s = p^{a+b}$, we have $p^{a+b} \mid (x \cdot y)$, hence $v_p(x \cdot y) \ge a + b$ by definition of valuation.

\smallskip
\noindent\textbf{Upper bound: $v_p(x \cdot y) \le a + b$.}

Suppose for contradiction that $v_p(x \cdot y) \ge a + b + 1$. Then
\[
x \cdot y = p^{a+b+1} \cdot z
\]
for some $z \in M$.

Consider the two-factorization $(p^{a+b+1}, z) \in \mathcal{F}_2(x \cdot y)$. Since $x \cdot y = s \cdot t$ with $s, t$ coprime, by \textup{(CFI)} this factorization must arise from some
\[
(s_1, s_2) \in \mathcal{F}_2(s) \quad \text{and} \quad (t_1, t_2) \in \mathcal{F}_2(t)
\]
via the coordinatewise map:
\[
p^{a+b+1} = s_1 \cdot t_1, \qquad z = s_2 \cdot t_2.
\]

By Lemma~\ref{lem:pp-unique}, since $s = p^{a+b}$, we have $s_1 = p^i$ and $s_2 = p^{a+b-i}$ for some $0 \le i \le a + b$.

Now, $p^{a+b+1} = s_1 \cdot t_1 = p^i \cdot t_1$. By \textup{(PP-P)}, since this product is a power of $p$, both $p^i$ and $t_1$ must be powers of $p$.

But $t_1$ is a factor of $t = x' \cdot y'$, which satisfies $v_p(t) = 0$. By \textup{(PP-P)}, any factor of $t$ must also satisfy $v_p(\cdot) = 0$ (since if a factor were divisible by $p$, then $t$ would be divisible by $p$ by Lemma~\ref{lem:pp-unique}). Therefore $t_1 = p^0 = 1$.

This gives $p^{a+b+1} = p^i \cdot 1 = p^i$, so by \textup{(PP-D)}, $a + b + 1 = i$. But $i \le a + b$ from above, yielding $a + b + 1 \le a + b$, a contradiction.

\smallskip
Therefore $v_p(x \cdot y) = a + b = v_p(x) + v_p(y)$.
\end{proof}

\begin{corollary}[Factorial structure]\label{cor:factorial}
Under \textup{(PP-D)} and \textup{(CFI)}, the map
\[
\Phi : \bigoplus_{p \in P} \Nzero \longrightarrow M, \qquad (e_p)_p \mapsto \prod_p p^{e_p}
\]
is a monoid isomorphism. In particular, $M$ is factorial, and every $m$ admits a unique factorization $m = \prod_p p^{v_p(m)}$.
\end{corollary}

\begin{proof}
Surjectivity follows from atomicity and Lemma~\ref{lem:primewise}. For injectivity, if $\prod_p p^{a_p} = \prod_p p^{b_p}$, apply $v_q$ (which is a monoid homomorphism by Proposition~\ref{prop:val-additive}) to get $a_q = b_q$ for all $q$.
\end{proof}

\begin{remark}[Cancellativity]\label{rem:cancellativity}
Corollary~\ref{cor:factorial} implies that $M$ is cancellative: if $a \cdot b = a \cdot c$, then applying $v_p$ for each $p$ gives $v_p(b) = v_p(c)$, hence $b = c$ by the unique representation. Thus cancellativity is a \emph{consequence} of our axioms, not an assumption.
\end{remark}

%=============================================================================
\section{Main Theorem}\label{sec:main}
%=============================================================================

In this section we combine the local analysis (\textup{PP-D} and the derived \textup{PP-P}) with the global independence (\textup{CFI}) and the unbounded coprimeness axiom (\textup{CPL}) to identify $M$ with $(\N, \times)$ up to relabeling.

\begin{theorem}[Main result]\label{thm:main}
Let $M$ be a reduced atomic commutative monoid with atom set $P$. Assume:
\begin{enumerate}[label=\textup{(\roman*)}]
\item \textup{(PP-D)}: Powers of each atom are distinct.
\item \textup{(CFI)}: Coprime parts factor independently.
\item \textup{(CPL)}: Pairwise coprime tuples of non-units exist in every length.
\end{enumerate}
Then:
\begin{enumerate}[label=\textup{(\alph*)}]
\item $M$ is factorial, canonically isomorphic to the free commutative monoid $\bigoplus_{p \in P} \Nzero$. Every $m \in M$ has a unique factorization $m = \prod_{p \in P} p^{v_p(m)}$.
\item The atom set $P$ is countably infinite. Hence there exists a bijection $\iota : P \to \PP$ such that
\[
\Psi : M \longrightarrow (\N, \times), \qquad \Psi\Bigl(\prod_{p \in P} p^{e_p}\Bigr) = \prod_{p \in P} \iota(p)^{e_p}
\]
is a monoid isomorphism. Equivalently, after relabeling atoms via $\iota$, the monoid operation on $M$ is ordinary multiplication and $P$ is the set of ordinary primes.
\end{enumerate}
\end{theorem}

\begin{proof}
By Proposition~\ref{prop:CFI-implies-PPP}, the base assumptions together with \textup{(CFI)} imply \textup{(PP-P)} (each prime fiber $\langle p \rangle$ is factorially closed). Then Theorem~\ref{thm:local-sb} (with \textup{(PP-D)} and \textup{(PP-P)}) gives the local stars-and-bars formula $F_k(p^e) = \binom{e+k-1}{k-1}$. By Lemma~\ref{lem:CFI-k} and Proposition~\ref{prop:coprime-mult}, \textup{(CFI)} yields coprime multiplicativity of the $k$-fold counts, hence Theorem~\ref{thm:master} applies and gives the global master formula $F_k(m) = \prod_{p \in P} \binom{v_p(m)+k-1}{k-1}$. From the master formula we obtain additivity of valuations (Proposition~\ref{prop:val-additive}): $v_p(x \cdot y) = v_p(x) + v_p(y)$ for all $x, y$ and $p \in P$. Therefore Corollary~\ref{cor:factorial} applies: the map $\Phi : (e_p)_p \mapsto \prod_p p^{e_p}$ is a monoid isomorphism $\bigoplus_{p \in P} \Nzero \xrightarrow{\sim} M$, proving (a).

For (b), \textup{(CPL)} together with the base implies that $P$ is infinite: given $r$, take pairwise coprime non-units $m_1, \ldots, m_r$; choosing an atom $p_i \mid m_i$ shows that $p_1, \ldots, p_r$ are distinct, hence $|P| \ge r$. Thus $P$ is infinite (indeed countable, since $M$ is countable). Choosing any bijection $\iota : P \to \PP$ with the ordinary primes, the map $\Psi$ in the statement is a monoid isomorphism, so the monoid operation is ordinary multiplication after relabeling and $P$ identifies with $\PP$.
\end{proof}

%=============================================================================
\section{Sharpness: Independence of the Axioms}\label{sec:sharpness}
%=============================================================================

Each of the three axioms in Theorem~\ref{thm:main} is necessary: below we construct explicit reduced atomic commutative monoids satisfying two of the axioms while violating the third.

\begin{example}[Failure of PP-D only]\label{ex:fail-PPD}
Fix an integer $B \ge 2$. Let $N$ be the commutative monoid generated by a countable set $\mathcal{G} = \{g_1, g_2, g_3, \ldots\}$ subject to the single relation $g_1^{B+1} = g_1^B$, and no others. Thus $g_2, g_3, \ldots$ behave freely (ordinary addition of exponents), while the $g_1$-tower \emph{saturates} at level $B$ (so $g_1^e = g_1^B$ for all $e \ge B$).

Choose a bijection $\psi : \N \to N$ with $\psi(1) = 1_N$ and, for concreteness, $\psi(p_i) = g_i$ on the $i$-th usual prime $p_i$. Define a monoid structure on $\N$ by
\[
n \star m := \psi^{-1}\bigl(\psi(n) \cdot \psi(m)\bigr), \qquad P := \psi^{-1}(\mathcal{G}).
\]
Then $(\N, \star, 1)$ is a reduced atomic commutative monoid (every $n$ corresponds to a finite word in the generators). Moreover:

\begin{itemize}[leftmargin=1.6em]
\item \textup{(CFI)} holds: if $x, y$ are $\star$-coprime (their $\psi$-supports are disjoint), then in $N$ factorization across disjoint generators is componentwise; the coordinatewise map $\mathcal{F}_2(x) \times \mathcal{F}_2(y) \to \mathcal{F}_2(x \star y)$ is bijective, and hence so in $(\N, \star)$.

\item \textup{(PP-D)} \emph{fails}: along the atom $p_1 := \psi^{-1}(g_1)$ we have
\[
p_1^{\star(B+1)} = \psi^{-1}(g_1^{B+1}) = \psi^{-1}(g_1^B) = p_1^{\star B},
\]
so powers of $p_1$ are \emph{not} all distinct.

\item \textup{(CPL)} holds: for each $r \in \N$, the elements $\psi^{-1}(g_2), \psi^{-1}(g_3), \ldots, \psi^{-1}(g_{r+1})$ are non-units with pairwise disjoint $\psi$-supports, hence are pairwise $\star$-coprime.
\end{itemize}

Intuitively, this is ordinary prime-power arithmetic on all generators except that the $p_1$-tower saturates past level $B$. Because we defined $\star$ by transporting the monoid law of $N$ via the bijection $\psi$, factorization exists for every $n \in \N$, and \textup{(CFI)} and \textup{(CPL)} remain true while \textup{(PP-D)} fails exactly at $p_1$.
\end{example}

\begin{example}[Failure of CFI only]\label{ex:fail-CFI}
Pick distinct usual primes $p, q, u, v$. Let $\sim$ be the smallest commutative-monoid congruence on $(\N, \times)$ generated by the single relation $uv \sim pq$. Choose a bijection $\varphi : \N \to \N/{\sim}$ with $\varphi(1) = [1]$ and define
\[
n \star m := \varphi^{-1}\bigl(\varphi(n) \cdot \varphi(m)\bigr), \qquad P := \varphi^{-1}\bigl(\{[r] : r \text{ a usual prime}\}\bigr).
\]
Then $(\N, \star, 1)$ is a reduced atomic commutative monoid (every element of $\N/{\sim}$ is a finite product of the classes $[r]$, hence so is every $n \in \N$ via $\varphi^{-1}$). Moreover:

\begin{itemize}[leftmargin=1.6em]
\item \textup{(PP-D)} holds: The relation $uv \sim pq$ does not identify distinct pure powers of the \emph{same} prime class, so powers in each $\langle \varphi^{-1}([r]) \rangle$ are distinct.

\item \textup{(CFI)} \emph{fails}: Take $x := \varphi^{-1}([p])$ and $y := \varphi^{-1}([q])$, which are $\star$-coprime. Then $x \star y = \varphi^{-1}([p][q]) = \varphi^{-1}([u][v])$, so $\bigl(\varphi^{-1}([u]), \varphi^{-1}([v])\bigr) \in \mathcal{F}_2(x \star y)$ is a two-factorization \emph{not} in the image of the coordinatewise map $\mu_2 : \mathcal{F}_2(x) \times \mathcal{F}_2(y) \to \mathcal{F}_2(x \star y)$ (whose image consists only of $\bigl(\varphi^{-1}([p]), \varphi^{-1}([q])\bigr)$ and $\bigl(\varphi^{-1}([q]), \varphi^{-1}([p])\bigr)$). Hence $\mu_2$ is not surjective and \textup{(CFI)} fails.

\item \textup{(CPL)} holds: For each $r \in \N$, choose distinct prime classes $[t_1], \ldots, [t_r]$ and set $n_i := \varphi^{-1}([t_i])$. Then $n_i$ are non-units with pairwise disjoint $\star$-supports, hence are pairwise $\star$-coprime.
\end{itemize}
\end{example}

\begin{example}[Failure of CPL only]\label{ex:fail-CPL}
Consider the Peano monoid from Example~\ref{ex:peano}: $x \star y = x + y - 1$ on $\N$, with unique atom $P = \{2\}$. There $2$ is the unique atom and every $n \ge 2$ satisfies $n = 2^{\star(n-1)}$, so $(\N, \star, P)$ is a reduced atomic commutative monoid. Moreover:

\begin{itemize}[leftmargin=1.6em]
\item \textup{(PP-D)} holds: $2^{\star e} = e + 1$ is injective in $e$, so the $2$-powers are all distinct.

\item \textup{(CFI)} holds vacuously: the only coprime pairs involve the identity $1$, since every $n > 1$ is divisible by $2$; for such pairs the coordinatewise map is trivially bijective.

\item \textup{(CPL)} \emph{fails}: since every non-unit is divisible by the unique atom $2$, any two non-units share this factor and are not $\star$-coprime. We cannot find even $r = 2$ pairwise coprime non-units.
\end{itemize}

This shows that \textup{(CPL)} is indispensable in Theorem~\ref{thm:main}: without it, rank-one free monoids satisfy \textup{(PP-D)} and \textup{(CFI)} but cannot be isomorphic to $(\N, \times)$ since $|P| = 1$ cannot be put in bijection with the countably infinite set of ordinary primes.
\end{example}

%=============================================================================
\section{Conclusion}\label{sec:conclusion}
%=============================================================================

We have shown that three transparent axioms---\textup{(PP-D)}, \textup{(CFI)}, and \textup{(CPL)}---characterize the multiplicative monoid $(\N, \times)$ among reduced atomic commutative monoids. The key innovation is the use of \emph{labeled $k$-factorization counts} $F_k$ and a \emph{bijection-level} coprimality condition \textup{(CFI)}, which together force the stars-and-bars formula and hence factorial structure.

Our approach complements the classical theory of non-unique factorization \cite{GeroldingerHalterKoch2006, GeroldingerZhong2020}: rather than measuring how far a monoid departs from being factorial, we give counting-based criteria that \emph{characterize} factorial monoids of countably infinite rank.

\subsection*{Open questions}

We conjecture that an analogous characterization can be obtained for \emph{unlabeled} factorizations, replacing stars-and-bars counts by partition numbers on at most $k$ parts and using a suitably symmetrized form of \textup{(CFI)}. We leave this direction for future work.

\section*{Acknowledgments}

During drafting and revision, I used ChatGPT 5 Thinking, ChatGPT 5 Pro, and Claude Sonnet 4.5 for brainstorming, phrasing alternatives, anticipating potential referee feedback, and light copy-editing.

\bigskip
\noindent\textbf{Data availability.} No datasets were generated or analyzed.

\smallskip
\noindent\textbf{Conflict of interest.} The author declares no conflict of interest.

%=============================================================================
\section*{Appendix: Reducing CFI to Prime-Power Pairs}
%=============================================================================

The following two lemmas record the inductive argument that reduces the full coprime-factorization bijection \textup{(CFI)} to its verification on pairs of prime powers, using only the base assumptions of a reduced atomic commutative monoid.

\begin{lemma}[CFI for one prime power against a product of distinct prime powers]\label{lem:app-CFIpp-vs-product}
Suppose \textup{(CFI)} holds for every pair of prime powers $(p^a, q^b)$ with $p \neq q$ and $a, b \ge 0$. Fix $p \in P$, $a \ge 0$, and write
\[
y = \prod_{j=1}^r q_j^{b_j} \qquad (r \ge 0),
\]
where $q_1, \ldots, q_r \in P \setminus \{p\}$ are pairwise distinct and $b_j \ge 0$. Then \textup{(CFI)} holds for the coprime pair $(p^a, y)$, i.e., the coordinatewise map
\[
\mu_2 : \mathcal{F}_2(p^a) \times \mathcal{F}_2(y) \xrightarrow{\;\cong\;} \mathcal{F}_2(p^a \cdot y)
\]
is a bijection. Moreover, the bijection is independent of the order in which the factors $q_j^{b_j}$ are glued in.
\end{lemma}

\begin{proof}
Induction on $r$. For $r = 0$ the statement is tautological. For $r = 1$ it is exactly the assumed $\textup{CFI}(p^a, q_1^{b_1})$.

Assume the claim for $r - 1 \ge 1$ and write $y = y' \cdot q_r^{b_r}$ with $y' = \prod_{j=1}^{r-1} q_j^{b_j}$. By the induction hypothesis we have a bijection
\[
\mathcal{F}_2(p^a) \times \mathcal{F}_2(y') \xrightarrow{\;\cong\;} \mathcal{F}_2(p^a \cdot y'). \tag{$*$}
\]
Since $q_r \neq p$ and $q_r \notin \{q_1, \ldots, q_{r-1}\}$, the pair $(p^a \cdot y', q_r^{b_r})$ is coprime, so by the assumed $\textup{CFI}(p^a \cdot y', q_r^{b_r})$ we have a bijection
\[
\mathcal{F}_2(p^a \cdot y') \times \mathcal{F}_2(q_r^{b_r}) \xrightarrow{\;\cong\;} \mathcal{F}_2(p^a \cdot y' \cdot q_r^{b_r}). \tag{$**$}
\]
Composing $(*)$ and $(**)$, and using associativity/commutativity to re-associate factors, gives the desired bijection $\mathcal{F}_2(p^a) \times \mathcal{F}_2(y) \to \mathcal{F}_2(p^a \cdot y)$. Order-independence follows because changing the order of gluing merely permutes the intermediate coordinates and reassociates products, which does not affect the coordinatewise image.
\end{proof}

\begin{lemma}[Prime-power testing suffices for full CFI]\label{lem:app-pp-testing}
Suppose \textup{(CFI)} holds for every pair of prime powers $(p^a, q^b)$ with $p \neq q$. Then \textup{(CFI)} holds for all coprime pairs $(x, y)$.
\end{lemma}

\begin{proof}
Write
\[
x = \prod_{i=1}^r p_i^{a_i}, \qquad y = \prod_{j=1}^s q_j^{b_j},
\]
with $p_1, \ldots, p_r, q_1, \ldots, q_s \in P$ pairwise distinct (disjoint supports) and $a_i, b_j \ge 0$. We proceed by induction on $r$. For $r = 0$ the claim is trivial; for $r = 1$ it is exactly Lemma~\ref{lem:app-CFIpp-vs-product} with $p = p_1$, $a = a_1$.

Assume the claim for $r - 1 \ge 1$ and set $x' := \prod_{i=2}^r p_i^{a_i}$, so $x = p_1^{a_1} \cdot x'$. By Lemma~\ref{lem:app-CFIpp-vs-product}, \textup{(CFI)} holds for $(p_1^{a_1}, y)$, giving a bijection
\[
\mathcal{F}_2(p_1^{a_1}) \times \mathcal{F}_2(y) \xrightarrow{\;\cong\;} \mathcal{F}_2(p_1^{a_1} \cdot y). \tag{$\dagger$}
\]
By the induction hypothesis, \textup{(CFI)} holds for $(x', y)$, giving
\[
\mathcal{F}_2(x') \times \mathcal{F}_2(y) \xrightarrow{\;\cong\;} \mathcal{F}_2(x' \cdot y). \tag{$\ddagger$}
\]
Composing $(\dagger)$ and $(\ddagger)$ coordinatewise (and re-associating) yields a bijection
\[
\bigl(\mathcal{F}_2(p_1^{a_1}) \times \mathcal{F}_2(x')\bigr) \times \mathcal{F}_2(y) \xrightarrow{\;\cong\;} \mathcal{F}_2(p_1^{a_1} \cdot x' \cdot y) = \mathcal{F}_2(x \cdot y).
\]
Identifying $\mathcal{F}_2(x) \cong \mathcal{F}_2(p_1^{a_1}) \times \mathcal{F}_2(x')$ (by concatenating the first coordinate with the product of the remaining ones) shows this is exactly the coordinatewise map $\mu_2$, proving \textup{(CFI)} for $(x, y)$.
\end{proof}

\begin{thebibliography}{99}

\bibitem{GeroldingerHalterKoch2006}
A.~Geroldinger and F.~Halter-Koch,
\newblock \emph{Non-Unique Factorizations: Algebraic, Combinatorial and Analytic Theory}.
\newblock Chapman \& Hall/CRC, 2006.

\bibitem{GeroldingerZhong2020}
A.~Geroldinger and Q.~Zhong,
\newblock Factorization theory in commutative monoids.
\newblock \emph{Semigroup Forum} \textbf{100} (2020), 22--51.

\bibitem{OrderedFact2016}
J.~Sprittulla,
\newblock Ordered factorizations with $k$ factors.
\newblock arXiv:1610.04826, 2016.

\bibitem{GoemansGenFunc}
M.~X.~Goemans,
\newblock Generating Functions (MIT 18.310 lecture notes), 2015.

\bibitem{ApostolANT}
T.~M.~Apostol,
\newblock \emph{Introduction to Analytic Number Theory}.
\newblock Springer, 1976.

\bibitem{Tenenbaum}
G.~Tenenbaum,
\newblock \emph{Introduction to Analytic and Probabilistic Number Theory} (3rd ed.).
\newblock AMS, 2015.

\bibitem{Mercer2017}
I.~Mercer,
\newblock Another Proof That There Are Infinitely Many Primes.
\newblock \emph{Amer.\ Math.\ Monthly} \textbf{124} (2017), 169.

\end{thebibliography}

\end{document}
